In recent decades, there have been significant developments and breakthroughs in technology.The use of a variety of  electronics devices; such as mobile
phones, laptops, tablets and home appliances has significantly increased due to
consumer demands.when such electronic devices are upgraded or replaced, the
chargers also need replacing even though they may still be in good working order. 
These unused devices end up in landfill because of improper recycling. This
is known as electronic waste (E-Waste). Worldwide 20 to 50 million tonnes of
E-Waste is annually generated.

On the other hand, over the last few years, fossil fuel prices have increased rapidly because of the reliance
on limited hydrocarbon energy sources (such as petrol, diesel and gas) for power
generation and advanced transportation systems. In order to reduce the usage of fossil
fuels, electrified transportation requires the set-up of a wide variety of charging networks
to create user-friendly environments and to tackle green-house and fuel price hikes.

To compete with gasoline-powered cars, EVs are required to have a long travelling range with an efficient refuel
capacity. These features can only be achieved by installing a larger battery bank with
continual charging facilities. However, a larger battery bank increases the overall price
of EVs and requires additional charging time. Such limitations are the biggest hurdles in
making EVs a reliable transportation alternative.

In order to tackle these issues, Wireless Charging Systems (WCS) are one of the up and
coming technologies which have the potential to provide contactless charging facilities in all types of EVs.
WCS technologies  can offer safe, secure and user friendly charging
techniques that are environmentally sound. WCS can provide a common charging
platform for most Electric vehicles so different EVs do not require
separate standardised chargers to charge their batteries.Also,this technology can reduce toxic and non-biodegradable hazardous E-waste
by establishing a common charging platform for all electronic products.

But the problem or limitation associated with WCS is that they can only be utilized when the
car is parked or in stationary modes such as in car parks, garages or at traffic signals.
In addition, stationary WCS have some challenges such as
 limited power transfer, bulky structures, shorter ranges and higher
 efficiency levels. In order to improve: range and battery storage volume, the
 dynamic mode of operation of the WCS for EVs is developing in the world.





