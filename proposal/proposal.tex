

\documentclass[12pt]{article} 
\usepackage[utf8]{inputenc} % set input encoding (not needed with XeLaTeX)
\usepackage{cite}
\usepackage{geometry} \geometry{a4paper} 
\usepackage{graphicx} 
% \usepackage[parfill]{parskip} 

\begin{document}
\begin{titlepage}
	\newcommand{\HRule}{\rule{\linewidth}{1mm}} % Defines a new command for the horizontal lines, change thickness here
	\center % Center everything on the page
	%----------------------------------------------------------------------------------------
	%	H\begin{aligned}
	%----------------------------------------------------------------------------------------
	\includegraphics[scale=0.5]{logo.jpg}\\[1cm] % Include a department/university logo - this will require the graphicx package
	\textsc{\LARGE Tribhuvan University, }\\[0.4cm] % Name of your university/college
	\textsc{\Large Pulchowk Campus,}\\[0.4cm] % Major heading such as course name
	\textsc{\large Pulchowk,Lalitpur}\\[0.4cm] % Minor heading such as course title
	%----------------------------------------------------------------------------------------
	%	TITLE SECTION
	%----------------------------------------------------------------------------------------
	\HRule \\[0.4cm]
	{ \large \bfseries Efficiency Enhancement of Wireless Charging System \\[0.3cm] for EV in Static and Dynamic Mode} \\[0.4cm] % Title of your document
	\HRule \\[1.5cm]

	%----------------------------------------------------------------------------------------
	%	AUTHOR SECTION
	%----------------------------------------------------------------------------------------

	\begin{minipage}{0.4\textwidth}
		\begin{flushleft} 
			\emph{Submitted by:}\\
			Rajib Bijukchhe, 073BEL330\\
			Anjil Adhikari, 073BEL307\\
			Praveen Kushwaha, 073BEL328\\
			Rabin Dhamala, 073BEL329\\
		\end{flushleft}
	\end{minipage}
	~	
	\begin{minipage}{0.4\textwidth}
		\begin{flushright} 
			\emph{Submitted to:} \\
			Electrical Engineering Department \\
		\end{flushright}
	\end{minipage}\\[2cm]

	% If you don't want a supervisor, uncomment the two lines below and remove the section above
	%\Large \emph{Author:}\\
	%John \textsc{Smith}\\[3cm] % Your name

	%----------------------------------------------------------------------------------------
	%	DATE SECTION
	%----------------------------------------------------------------------------------------

	{\large \today}\\[2cm] % Date, change the \today to a set date if you want to be precise

	%----------------------------------------------------------------------------------------
	%	LOGO SECTION
	%----------------------------------------------------------------------------------------


	%----------------------------------------------------------------------------------------

	\vfill % Fill the rest of the page with whitespace
\end{titlepage}


\tableofcontents
\newpage

\renewcommand{\baselinestretch}{1.5}
\section{Objective}
\textit{To improve power transfer efficiency of wireless charging system for EVs in static and dynamic mode}

\section{Motivation}
In recent decades, there have been significant developments and breakthroughs intechnology. The use of a variety of autonomous electronics devices; such as mobile phones, laptops, tablets and home appliances has significantly increased due to consumer demands. Such electronics contain on-board energy storage modules in the form of lithium based batteries. These batteries have normally been charged by conventional battery chargers through physical contacts. This form of connection increases the risk of short circuits; either by water or foreign objects crossing the contacts, or due to damaged connectors. In addition, different electronic devices have different power ratings as well as a variety of standardized chargers. Standardised chargers create inconvenience to the users when they travel to another country. Moreover, when such electronic devices are upgraded or replaced, the chargers also need replacing even though they may still be in good working order. These unused devices end up in landfill because of improper recycling. This is known as electronic waste (E-Waste). Worldwide 20 to 50 million tonnes of E-Waste is annually generated.

In order to reduce the usage of high priced fossil fuels, electrified transportation requires the set-up of a wide variety of charging networks to create user-friendly environments and to tackle green-house and fuel price hikes. To compete with gasoline-powered cars, EVs are required to have a long travelling range with an efficient refuel capacity. These features can only be achieved by installing a larger battery bank with continual charging facilities. However, a larger battery bank increases the overall price of EVs and requires additional charging time. Such limitations are the biggest hurdles in making EVs a reliable transportation alternative. 

In order to tackle these issues, Wireless Charging Systems(WCS) are one of the up and coming technologies which have the potential to provide contactless charging facilities in low, medium and high power consumer electronic products. WCS technology can offer \textbf{safe, secure ,reliable} and \textbf{user friendly} charging techniques that are environmentally sound. WCS can provide a common charging platform for most electronic devices so different electronic products do not require separate standardised chargers to charge their batteries. Moreover, it can help to reduce the battery storage requirements by providing a continuous dynamic charging facility wirelessly. This technology can reduce toxic and non-biodegradable hazardous E-waste by establishing a common charging platform for all electronic products. %cite

\section{Introduction}
In the 21st century, wireless power transmissionsystems (WPTS) are considered to be advanced technologies for numerous applications
such as mobile devices, home appliances, and medical implants in the areas of low and medium voltage electronics devices, by the utilisation of two fundamental important mechanisms: inductive coupling and strong electromagnetic resonant coupling %[15].

Firstly, the inductive coupling method with resonant case was identified and used in the experiment by Nikola Tesla in 1914.  This method is also known as Resonant Inductive Coupling (RIC). His idea was to transfer power over a long distance globally. Radiative transfer is suitable for transferring data and other information over the air, over a long distance, by using antennas. However, when it comes to power transfer, it is very difficult to use this method because most of the energy is lost in the air because of the omni-directional transmission. Otherwise, it requires a line of sight and an advanced tracking system at the receiver side. Inductive coupling is an immersing method to transfer power wirelessly in the power ranges between mill watts and some kilowatts.
However, when it comes to efficiency, it decreases significantly with increasing distance between the primary and secondary circuit.The problem or limitation associated with WCS is that they can only be utilised when the vehicle is parked or in stationary modes such as in car parks, garages or at traffic signals . In addition, stationary WCS have some challenges such as limited power transfer, bulky structures,shorter range and higher efficiency. 

In order to improve both areas (range and sufficient volume of battery storage), a dynamic mode of operation of the WCS for EVs has been researched. Inductive charging of electric vehicle at high power levels while in motion is referred to as dynamic charging. If a car was able to be charged while it was being driven, then this would solve the problem of limited range and enable vehicles to travel for potentially unlimited distances. The idea is that a coil on the bottom of a car could receive electricity from coils that are connected to an electric current that is emnedded in the road. The vehicle would also requireless volume of expensive battery storage which increases the range of the transportation. A dynamic WCS can be significantly improved using , different controllers (PI controller, fuzzy, neuro-fuzzy MPPT) methods.\cite{knupfer2017electrifying} 
\newpage
A simple block diagram implementation of wireless charging system is shown below in figure \ref{fig:scheme}:
\begin{figure}[h!]
	\centering
	\includegraphics[scale=0.5]{image1.jpg}
	\caption{Schematic of Wireless Charging System}
	\label{fig:scheme}
\end{figure}

\section{Methodology}
\subsection{For static mode}
\begin{enumerate}
	\item The different resonant circuit will be modeled . Simulation results will be carried out showing the effect of parameters such as inductor, capacitor, load and coupling coefficient on efficiency. 
	\item 3D modeling of transmitting and receiving coil using ANSYS Maxwell. 
	\item Design of system circuit in ANSYS Simplorer. 
	\item Design of complete system of a magnetic resonance WPT in Maxwell. 
	\item Comparing Magnetic Induction Power Transfer with Magnetic Resonance Method. 
\end{enumerate}

\subsection{For dynamic mode}
\begin{enumerate}
	\item Design of three-level cascaded PI controller  to eliminate the variation of voltage because of varied spacing existing between both coils.
	\item Design of fuzzy logic and neuro-fuzzy controller.
	\item Design of three-level cascaded MPPT controller. 
		The design of dynamic mode controllers will be done using Matlab Simulink. 
\end{enumerate}
\section{Literature}
\begin{description}
	\item[Maximum power point tracking (MPPT)] or sometimes just power point tracking (PPT) is a technique  of finding point in which power transfer at highest efficiency. MPPT is used  to maximize power extraction under all conditions. Although it primarily applies to solar power, the principle applies generally to sources with variable power.MPPT devices are typically integrated into an electric power converter system that provides voltage or current conversion, filtering, and regulation for driving various loads, including power grids, batteries, or motors.
	\item[PID controller] A proportional-integral-derivative controller (PID controller or three-term controller) is a control loop mechanism employing feedback that is widely used in industrial control systems and a variety of other applications requiring continuously modulated control. A PID controller continuously calculates an error value e(t). If we put Proportional and Integral action together, we get the humble PI controller.
	\item[Fuzzy logic] Fuzzy logic is a form of many-valued logic in which the truth values of variables may be any real number between 0 and 1 both inclusive. It is employed to handle the concept of partial truth, where the truth value may range between completely true and completely false. By contrast, in Boolean logic, the truth values of variables may only be the integer values 0 or 1. A fuzzy control system is a control system based on fuzzy logic.
	\item[Neuro-Fuzzy logic] Neuro-fuzzy refers to combinations of artificial neural networks and fuzzy logic.Neuro-fuzzy hybridization results in a hybrid intelligent system that synergizes these two techniques by combining the human-like reasoning style of fuzzy systems with the learning and connectionist structure of neural networks. A neuro-fuzzy control system is a control system based on neuro-fuzzy logic.
\end{description}

\newpage

\section{Terms}
\textit{Wireless Charging Systems (WCS), 
	Wireless Power Transmission Systems (WPTS), 
	Resonant Inductive Coupling (RIC),
	Electric Vehicles (EV), 
	Maximum Power Point Tracking(MPPT),
	Wireless power transfer (WPT),
	Proportional Integral(PI),
	Proportional Integral Derivative(PID),
	Battery Management Systems (BMS),
Wireless Electric Vehicle Charging System (WEVCS)}
\section{Operating Theory}
\begin{figure}[h]
	\centering
	\includegraphics[scale =0.75]{static.png}
	\caption{Wireless Charging model of EV}
	\label{fig:WEV}
\end{figure}
Figure \ref{fig:WEV} shows the basic arrangement of WEVCS. The primary coil is installed underneath in the road or ground with additional power converters and circuitry. The receiver coil, or secondary coil, is normally installed underneath of the EVs front, back
or center. The receiving energy is converted from AC to DC using the power converter and is transferred to the battery bank. In order to avoid any safety issues, power control and battery management systems (BMS) are fitted with a wireless communication network to receive any feedback from the primary side. The charging time depends on the source power level, charging pad sizes, and air gap distance between the two windings. The average distance between lightweight duty vehicles is approximately 150 mm - 300 mm. Static WEVCS can be installed in parking areas, car parks, homes, commercial buildings, shopping centres,
  %block diagram of static
  
\bibliographystyle{plain}
\bibliography{main}
\end{document}
