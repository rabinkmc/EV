As more people and goods move around the planet, our cars, planes, trains, and
ships are having a growing impact on the climate. Transportation now generates
almost a quarter of the world's greenhouse gas emissions. Shifting from vehicles
that burn fossil fuels to those that run on electricity will play a key role in curtailing
climate change. 

Over the last few years, fossil fuel prices have increased rapidly because of the reliance
on limited hydrocarbon energy sources (such as petrol, diesel and gas) for power
generation and advanced transportation systems. In order to reduce the usage of fossil
fuels, electrified transportation requires the set-up of a wide variety of charging networks
to create user-friendly environments and to tackle green-house and fuel price hikes.

To compete with gasolinepowered cars, EVs are required to have a long travelling range with an efficient refuel
capacity. These features can only be achieved by installing a larger battery bank with
continual charging facilities. However, a larger battery bank increases the overall price
of EVs and requires additional charging time. Such limitations are the biggest hurdles in
making EVs a reliable transportation alternative. 

Here wireless charging looks promishing as it will reduce overall number of chargers
needed and reduce the on-board battery storage capacity by providing continuous
dynamic charging resulting in less E-waste generated globally. Moreover, it can help to
reduce the battery storage requirements by providing a continuous dynamic charg-
ing facility wirelessly. Also, WCS has many other advantages over traditional contact 
charging like simple in operation, reliable, safe, robust and user friendly. WCS can provide
a common charging platform for most Electric vehicles so different EVs do not require separate standardised 
chargers to charge their batteries.

But the problem or limitation associated with WCS is that they can only be utilized when the
car is parked or in stationary modes such as in car parks, garages or at traffic signals.
In addition, stationary WCS have some challenges such as limited power transfer, bulky 
structures, shorter ranges and lower efficiency levels. In order to improve: range and battery storage volume, the
dynamic mode of operation of the WCS for EVs is developing in the world.




